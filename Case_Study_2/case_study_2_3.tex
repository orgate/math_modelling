% Copyright 2004 by Till Tantau <tantau@users.sourceforge.net>.
%
% In principle, this file can be redistributed and/or modified under
% the terms of the GNU Public License, version 2.
%
% However, this file is supposed to be a template to be modified
% for your own needs. For this reason, if you use this file as a
% template and not specifically distribute it as part of a another
% package/program, I grant the extra permission to freely copy and
% modify this file as you see fit and even to delete this copyright
% notice. 

\documentclass{beamer}

% There are many different themes available for Beamer. A comprehensive
% list with examples is given here:
% http://deic.uab.es/~iblanes/beamer_gallery/index_by_theme.html
% You can uncomment the themes below if you would like to use a different
% one:
%\usetheme{AnnArbor}
%\usetheme{Antibes}
%\usetheme{Bergen}
%\usetheme{Berkeley}
%\usetheme{Berlin}
%\usetheme{Boadilla}
%\usetheme{boxes}
%\usetheme{CambridgeUS}
%\usetheme{Copenhagen}
%\usetheme{Darmstadt}
%\usetheme{default}
%\usetheme{Frankfurt}
%\usetheme{Goettingen}
%\usetheme{Hannover}
%\usetheme{Ilmenau}
%\usetheme{JuanLesPins}
%\usetheme{Luebeck}
\usetheme{Madrid}
%\usetheme{Malmoe}
%\usetheme{Marburg}
%\usetheme{Montpellier}
%\usetheme{PaloAlto}
%\usetheme{Pittsburgh}
%\usetheme{Rochester}
%\usetheme{Singapore}
%\usetheme{Szeged}
%\usetheme{Warsaw}

\title{Case Study 2 and 3}

% A subtitle is optional and this may be deleted
\subtitle{Traffic flow modelling and Infectious disease modelling}

\author{Alfred Ajay Aureate R\inst{1}}
% - Give the names in the same order as the appear in the paper.
% - Use the \inst{?} command only if the authors have different
%   affiliation.

\institute[IITM] % (optional, but mostly needed)
{
  \inst{1}%
  EE10B052\\
  Department of Electrical Engineering\\
  Indian Institute of Technology Madras}
% - Use the \inst command only if there are several affiliations.
% - Keep it simple, no one is interested in your street address.

\date{MA5710, Dec 2014}
% - Either use conference name or its abbreviation.
% - Not really informative to the audience, more for people (including
%   yourself) who are reading the slides online

\subject{Mathematics}
% This is only inserted into the PDF information catalog. Can be left
% out. 

% If you have a file called "university-logo-filename.xxx", where xxx
% is a graphic format that can be processed by latex or pdflatex,
% resp., then you can add a logo as follows:

% \pgfdeclareimage[height=0.5cm]{university-logo}{university-logo-filename}
% \logo{\pgfuseimage{university-logo}}

% Delete this, if you do not want the table of contents to pop up at
% the beginning of each subsection:
\AtBeginSubsection[]
{
  \begin{frame}<beamer>{Outline}
    \tableofcontents[currentsection,currentsubsection]
  \end{frame}
}

% Let's get started
\begin{document}

\begin{frame}
  \titlepage
\end{frame}

\begin{frame}{Outline}
  \tableofcontents
  % You might wish to add the option [pausesections]
\end{frame}

% Section and subsections will appear in the presentation overview
% and table of contents.
\section{Trafic flow modelling}

\subsection{Linear scalar conservation laws}

\begin{frame}{Linear scalar conservation law description}
  \begin{itemize}
  \item {
    $\begin{array}{cc}
u_{t}+au_{x}=0, & x\in\mathbb{R},t>0,\\
u\left(x,0\right)=u_{0}\left(x\right), & x\in\mathbb{R},
\end{array}$
where $a>0$.
  }
  \item {
    Now, we discretize the (x,t) plane.
    $\begin{array}{ccccc}
x_{i}=ih & \left(i\in\mathbb{Z}\right), & t_{n}=nk & \left(n\in\mathbb{N}_{0}\right) & h,k>0\end{array}$
    For simplicity, we take a uniform mesh with $h$ and $k$ constant. 
  }
  \end{itemize}
\end{frame}

% You can reveal the parts of a slide one at a time
% with the \pause command:
\begin{frame}{Central difference scheme}
  \begin{itemize}
  \item { Scalar conservation laws, when central differnce scheme:
    $\begin{array}{cc}
\frac{u_{i}^{n+1}-u_{i}^{n}}{k}=-a\frac{u_{i+1}^{n}-u_{i}^{n}}{2h}, & n\geq0,i\in\mathbb{Z}\end{array}$
    \pause % The slide will pause after showing the first item
  }
  \item {   
    It can be rewritten as: $u_{i}^{n+1}=u_{i}^{n}-\frac{ak}{2h}\left(u_{i+1}^{n}-u_{i-1}^{n}\right)$. This is explicit scheme.
    \pause
  }
  % You can also specify when the content should appear
  % by using <n->:
  \item<3-> {
    It can also be rewritten into implicit form: $\frac{ak}{2h}u_{i+1}^{n+1}+u_{i}^{n+1}-\frac{ak}{2h}u_{i-1}^{n+1}=u_{i}^{n}$
  }
  \end{itemize}
\end{frame}

\begin{frame}{Boundary conditions}
\begin{itemize}
\item {Periodic boundary conditions: $\begin{array}{cc}
u\left(0,t\right)=u\left(N,h,t\right), & t>0\end{array}$
}
\item {Discretized version: $\begin{array}{cc}
u_{0}^{n}=u_{N}^{n}, & n\geq0\end{array}$}
\item {Also by periodicity, we need to assume the following to use the schemes: $u_{-1}^{n}=u_{N-1}^{n}$ and $u_{N}^{n}=u_{0}^{N}$}
\item {For continuous case: $\begin{array}{cc}
u_{0}\left(x\right)=\sin\left(2\pi x\right), & 0\leq x\leq1\end{array}$
}
\item {For discontinuous case: $u_{0}\left(x\right)=\begin{cases}
\begin{array}{c}
1:\\
0:
\end{array} & \begin{array}{c}
0\leq x<1/2\\
1/2\leq x\leq1
\end{array}\end{cases}$}
\item {h = 0.01 and x = 0 to 1

k = 0.001 and t = 0 to 0.25}
\end{itemize}
\end{frame}

\begin{frame}{Lax Friedrich scheme}
\begin{itemize}
\item {Analytically, it could be written as: $\frac{1}{k}\left(u\left(x,t+k\right)-\frac{1}{2}\left(u\left(x+h,t\right)+u\left(x-h,t\right)\right)\right)$
}
\item {Here, the spatial derivative is approximated by the central difference scheme and thus the discretized version looks like: $u_{i}^{n+1}=\frac{1}{2}\left(u_{i+1}^{n}+u_{i-1}^{n}\right)-\frac{ak}{2h}\left(u_{i+1}^{n}-u_{i-1}^{n}\right)$}
\end{itemize}
\end{frame}

\begin{frame}{Downwind and Upwind scheme}
\begin{itemize}
\item {In downwind scheme: $u_{i}^{n+1}=u_{i}^{n}-\frac{ak}{2h}\left(u_{i+1}^{n}-u_{i}^{n}\right)$}
\item {This scheme is unstable}
\item {In upwind scheme: $u_{i}^{n+1}=u_{i}^{n}-\frac{ak}{2h}\left(u_{i}^{n}-u_{i-1}^{n}\right)$}
\item {This scheme is almost exact.}
\end{itemize}
\end{frame}

\begin{frame}{LWR model for single lane traffic simulation}
\begin{itemize}
\item {LWR model for single lane: $\begin{array}{cc}
u_{t}+f\left(u\right)_{x}=0, & x\in\mathbb{R},t>0,\\
u\left(x,0\right)=u_{0}\left(x\right), & x\mathbb{\in R},f:\mathbb{\mathbb{R}\rightarrow R}.\\
f\left(\rho\left(x,t\right)\right)\equiv u_{max}\rho\left(1-\frac{\rho}{\rho_{max}}\right)
\end{array}$}
\end{itemize}
\end{frame}

\begin{frame}{LWR model for single lane with traffic jam simulation}
\begin{itemize}
\item {LWR model for single lane: $\begin{array}{cc}
u_{t}+f\left(u\right)_{x}=0, & x\in\mathbb{R},t>0,\\
u\left(x,0\right)=u_{0}\left(x\right), & x\mathbb{\in R},f:\mathbb{\mathbb{R}\rightarrow R}.\\
f\left(\rho\left(x,t\right)\right)\equiv u_{max}\rho\left(1-\frac{\rho}{\rho_{max}}\right)
\end{array}$}
\end{itemize}
\end{frame}

\section{Infectious disease modelling}

\subsection{Procedure}

\begin{frame}{Necessary things}
\begin{itemize}
\item {Start with a base model}
\item {Parameter identification}
\item {Sensitivity of the parameters}
\item {Possibility of the solution}
\item {Stability of the system}
\item {Otherwise Lyapunov function's existence}
\end{itemize}
\end{frame}

\begin{frame}{Lyapunov function}
\begin{itemize}
\item {The equations describing a system could be written as: $\frac{d}{dt}\overline{X}\left(t\right)=A\overline{X}\left(t\right)$
and the initial conditions are given by: $\overline{X}\left(0\right)=\overline{X}_{0}$}
\item {Now, here if $Re\left(eigenvalue\left(A\right)\right)<0$,  for all eigen values of $A$, then the system is stable, otherwise it is not stable.}
\item {One more way of confirming the stability without knowing the solutions of the system, is by finding the Lyapunov function associated with the system. If a Lyapunov function exists, then the system is stable, otherwise, it is not. But, finding Lyapunov function for a system is not straightforward.}
\end{itemize}

\end{frame}
\begin{frame}{Lyapunov function}
Let V: $\mathbb{R}^{n}\rightarrow\mathbb{R}$
 

be a continuous scalar function.

V
  is a Lyapunov-candidate-function if it is locally positive-definite, i.e.

$V\left(0\right)=0$
 

$V\left(x\right)>0$
  $\forall x\in U{0}$
 

with U
  being a neighbourhood region around x=0
 .

Let g:$\mathbb{R}^{n}\rightarrow\mathbb{R}^{n}
 ;$ $\overset{.}{y}=g\left(y\right);$
 

be an arbitrary autonomous dynamical system with equillibrium point: y*
 :

$0=g\left(y*\right)$.
 

There always exists a coordinate transformation x=y-y*
 , such that:

$\overset{.}{x}=\overset{.}{y=g\left(y\right)=g\left(x+y*\right)=f\left(x\right)}$
 

$f\left(0\right)=0$
 .

So the new system $f\left(x\right)$
 has an equilibrium point at the origin.
\end{frame}

\begin{frame}{Lyapunov function}

Let x*=0
  be an equilibrium of the autonomous system $\overset{.}{x}=f\left(x\right)$
 .

And let $\overset{.}{V}\left(x\right)=\frac{d}{dt}V\left(x\left(t\right)\right)=\frac{\partial V}{\partial x}.\frac{dx}{dt}=\nabla V.\overset{.}{x}=\nabla V.f\left(x\right)$
 

be the time derivative of the Lyapunov-candidate-function V
 .

If the Lyapunov-candidate-function V
  is locally positive definite and the time derivative of the Lyapunov-candidate-function is locally negative semidefinite:

$\overset{.}{V}\left(x\right)\leq0
  \forall x\in\mathcal{B}{0}$
 

for some neighborhood $\mathcal{B}$
  of 0
 , then the equilibrium is proven to be stable.

If the Lyapunov-candidate-function V
  is locally positive definite and the time derivative of the Lyapunov-candidate-function is locally negative definite:

$\dot{V}(x)<0\quad\forall x\in\mathcal{B}\setminus\{0\}$
 

for some neighborhood $\mathcal{B}$
  of 0
 , then the equilibrium is proven to be locally asymptotically stable.
\end{frame}

\begin{frame}{Lyapunov function}

If the Lyapunov-candidate-function V
  is globally positive definite, radially unbounded and the time derivative of the Lyapunov-candidate-function is globally negative definite:

$\dot{V}(x)<0\quad\forall x\in\mathbb{R}^{n}\setminus\{0\}$
 ,

then the equilibrium is proven to be globally asymptotically stable.

The Lyapunov-candidate function V(x)
  is radially unbounded if

$\|x\|\to\infty\Rightarrow V(x)\to\infty$
 . 

\end{frame}

\begin{frame}{System equations}
$\frac{dV}{dt}=\alpha V-pVF$
 

$\frac{dF}{dt}=\beta C-\gamma pVF-aF$
 

$\frac{dC}{dt}=-\mu\left(c-c*\right)+g\left(m\right)kV_{t-T}F_{t-T}$
 

$\frac{dm}{dt}=\sigma V-\eta m\left(t\right)$
 

V - antigen count

F - antibody count

C - plasma cells

m - relative characteristic of the organ

$\alpha$ - the rate of antigen multiplication

p - the rate of contact between antigen and antibodies

$\beta$ - the rate of antibody production by plasma cell
\end{frame}
\begin{frame}
$\gamma$ - amount of antibodies necessary for the neutralization of the antigens

a - mean lifetime of antibodies

$\mu$ - mean lifetime of plasma cells

k  - the coefficient of immune system response

$\sigma$ - the rate of injury by antigen

$\eta$ - the rate of regeneration of the mass of the damaged

T - time taken for antigens to show response and die

$C*$ - normal level of plasma cells

$g(m)$ - describes dysfunction of immune system due to substantial organ damage
\end{frame}

\end{document}



